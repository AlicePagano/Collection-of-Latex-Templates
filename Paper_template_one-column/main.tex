% Paper template for Advanced Physics Laboratory Course
% Suggestion: it is better to compile it directly on overleaf, because the subfigure settings can arise problems
% Made by Alice Pagano and Francesca Dodici

\documentclass[rmp,10pt,onecolumn,fleqn,notitlepage]{revtex4-1}

% LOAD PACKAGES

% GENERAL PACKAGES
\usepackage{graphicx}
\usepackage{wrapfig}
\usepackage{color}
\usepackage{latexsym,amsmath}
\usepackage{physics}
\usepackage{chemformula}
\usepackage{tabularx}
\usepackage{float}
\usepackage{siunitx}
\usepackage{amssymb}
\usepackage[caption=false]{subfig} % rovina la formattazione delle figure se caption=true


\usepackage{multirow}

% LISTING PACKAGES
\usepackage{xcolor}
\usepackage{listings}
\usepackage{framed}
\usepackage{inconsolata} % To change the listing font

% URL PACKAGE AND SETTING
\definecolor{linkcolor}{rgb}{0,0,0.65} %hyperlink
\definecolor{linescolor}{rgb}{0.65,0.16,0.16}
\definecolor{cool}{RGB}{49,54,149}
\definecolor{hot}{RGB}{165,0,38}
\usepackage[pdftex,colorlinks=true, pdfstartview=FitV, linkcolor= linescolor, citecolor= linescolor, urlcolor= linkcolor, hyperindex=true,hyperfigures=true]{hyperref} %hyperlink%

% PAGE SETTING
\usepackage{fancyhdr}

\pagestyle{fancyplain}
\fancyhf{}
\fancyfoot[R]{\textbf{\thepage}}
\fancyfoot[L]{Università degli Studi di Padova - Dipartimento di Fisica e Astronomia Galileo Galilei}
%\fancyhead[L]{\textbf{Advanced Physics Laboratory Report}}
\fancyhead[L]{\ifnum\value{section}>0\nouppercase{\textbf{\leftmark}\fi}}
\fancyhead[R]{\textbf{Name1 Surname1 - Name2 Surname2}}
\renewcommand{\headrulewidth}{0.2pt}
\renewcommand{\footrulewidth}{0.1pt}

\setlength\parindent{9pt} % To adjust the intendation

% SECTION STYLE
%Redefine \thesubsection as \thesection.\alph{subsection}. (\alph replaces the default \arabic; you could also choose, e.g., \Alph, \roman, and \Roman.)
\renewcommand{\thesection}{\textbf{\Roman{section}}}
\renewcommand{\thesubsection}{\textbf{\arabic{subsection}}}
\renewcommand{\thesubsubsection}{\textbf{\Alph{subsubsection}}}
%\renewcommand{\thesection}{\textbf{\arabic{section}}}
%\renewcommand{\thesubsection}{\thesection.\textbf{\arabic{subsection}}}
%\renewcommand{\thesubsubsection}{\textbf{\thesection.\arabic{subsection}.\arabic{subsubsection}}}

% FIGURE AND TABLE STYLE
\renewcommand{\tablename}{\textbf{TAB.}}
\renewcommand{\thetable}{\textbf{\arabic{table}}}

% Eliminare nella compilazione finale
%\renewcommand{\figurename}{\textbf{FIG.}}
%\renewcommand{\thefigure}{\textbf{\arabic{figure}}}

% BIBLIOGRAPHY FILE AND SETTING
\bibliographystyle{aipnum4-1}
\setcitestyle{numbers,square}



\begin{document}


% FRONTESPIZIO

% Insert logo images above
\begin{figure}[H]
\begin{minipage}{0.25\linewidth}
\includegraphics[width=\linewidth]{image/logo/logo_DFA.jpg}
\end{minipage}
\hfill
\begin{minipage}{0.35\linewidth}
\includegraphics[width=\textwidth]{image/logo/logo_800anni.png}
\end{minipage}
\end{figure}

% Make upper lines
\noindent\makebox[\linewidth]{\color{linescolor} \rule{0.85\paperwidth}{1.2 pt}}
\noindent\makebox[\linewidth]{\color{linescolor} \rule[0.3cm]{0.85\paperwidth}{1pt}}

% Title
\title{Insert title here}
\author{Name1 Surname1 - number1 - name1.surname1@studenti.unipd.it \\ Name2 Surname2 - number2 - name2.surname2@studenti.unipd.it}
\date{\today}

% Abstract
\begin{abstract}

Insert abstract here

\end{abstract}

% Make title
\maketitle

% Make lower line
\noindent\makebox[\linewidth]{\color{linescolor} \rule[0.1cm]{0.85\paperwidth}{1pt}}


% INDEX

% Index configuration
{
  \hypersetup{linkcolor=black}
  \tableofcontents
}
% Make lower lines
\noindent\makebox[\linewidth]{\color{linescolor} \rule[-0.2cm]{0.85\paperwidth}{1pt}}
\noindent\makebox[\linewidth]{\color{linescolor} \rule[0.3cm]{0.85\paperwidth}{1.2 pt}}
\pagebreak


% Page configuration: start counting page after the first one which is the title
\pagenumbering{gobble}
\newpage
\pagenumbering{arabic}






\section{Introduction}
\label{sec:introduction}

gidhgoirgor
jngineofg er

gdgfd

non funziona

In this section we describe...in FIG. \ref{fig:1_prova}.

	% To insert a single figure
	\begin{figure}[h!]
	    \centering
	    \includegraphics[width=0.7\textwidth]{image/prova.jpg}
	    \caption{Scheme of...}
	    \label{fig:1_prova}
	\end{figure}

	% To insert two subfigure
	\begin{figure}[H]
	\begin{minipage}[c]{0.49\linewidth}
	\subfloat[][Description of the subfigure]{\includegraphics[width=0.8\textwidth]{image/prova.jpg} \label{fig:2_prova_a} }
	\end{minipage}
	\begin{minipage}[]{0.49\linewidth}
	\centering
	\subfloat[][Description of the subfigure]{\includegraphics[width=0.8\textwidth]{image/prova.jpg}  \label{fig:2_prova_b} }
	\end{minipage}
	\caption{\label{fig:2_prova} In \ref{fig:2_prova_a} ..., while \ref{fig:2_prova_b} shows... questa è una caption di prova per vedere se l'allineamento lo fa giusto. Questa è una caption di prova per vedere se l'allineamento lo fa giusto. Questa è una caption di prova per vedere se l'allineamento lo fa giusto. Questa è una caption di prova per vedere se l'allineamento lo fa giusto.}
	\end{figure}
%
%	% To insert four subfigure
%	\begin{figure}[H]
%	\begin{minipage}[c]{0.49\linewidth}
%	\subfloat[][Description of the subfigure]{\includegraphics[width=0.8\textwidth]{image/prova.jpg} \label{fig:3_prova_a} }
%	\end{minipage}
%	\begin{minipage}[]{0.49\linewidth}
%	\centering
%	\subfloat[][Description of the subfigure]{\includegraphics[width=0.8\textwidth]{image/prova.jpg} \label{fig:3_prova_b} }
%	\end{minipage} \\
%	\vfill
%	\begin{minipage}[c]{0.49\linewidth}
%	\subfloat[][Description of the subfigure]{\includegraphics[width=0.8\textwidth]{image/prova.jpg} \label{fig:3_prova_c} }
%	\end{minipage}
%	\begin{minipage}[]{0.49\linewidth}
%	\centering
%	\subfloat[][Description of the subfigure]{\includegraphics[width=0.8\textwidth]{image/prova.jpg} \label{fig:3_prova_d} }
%	\end{minipage}
%	\caption{\label{fig:3_prova} In \ref{fig:3_prova_a} ..., \ref{fig:3_prova_b} shows..., \ref{fig:3_prova_c} and \ref{fig:3_prova_d}...  }
%	\end{figure}


	% Example to insert a well-formatted table
   	\begin{table}[H]
	\centering
   	 \sisetup{separate-uncertainty}
   	 \begin{tabular*}{\linewidth}{@{\extracolsep{\fill}}
   	 l
   	 l
   	 S[table-format=3(1)]  %intercept
   	 S[table-format=1.2(2)] %slope
   	 S[table-format=1.3]
   	 l
   	 l
  	 c
  	 c
  	}
        \toprule
    & \textbf{Linear fit} & \textbf{ Intercept [mV] }  & \textbf{ Slope [mV/K] } & $\pmb{r}$ & \textbf{$\pmb{\chi^{(2)}}$/ndf} & \textbf{p-value} & \textbf{$\pmb{T_c}$ [K]} & \textbf{$\pmb{V(T_c)}$ [mV]}\\
        \colrule
    \multirow{2}{*}{{\color{cool}\textbf{Cooling}}} & $\pmb{a_l + b_l \, x }$ &  453\pm3   &   -1.11\pm0.04 & -0.845 & 198.3/61 & $2 \cdot 10^{-16}$ & \multirow{2}{*}{ \tablenum{80\pm1}} & \multirow{2}{*}{ \tablenum{ 363\pm5} } \\
     & $\pmb{a_r + b_r \, x }$ &  99\pm22   &  3.3\pm0.3  & 0.896 & 90.36/17 & $5 \cdot 10^{-12}$\\
         \colrule
    \multirow{2}{*}{{\color{hot}\textbf{Heating}}} & $\pmb{a_l + b_l \, x }$ &  521 \pm 5   &   -1.19 \pm 0.05 & -0.940 & 54.72/57 & 0.561 & \multirow{2}{*}{ \tablenum{ 109 \pm 3}} & \multirow{2}{*}{ \tablenum{ 392\pm8} } \\
     & $\pmb{a_r + b_r \, x }$ &  173\pm10   &  2.01\pm0.09  & 0.915 & 47.81/33& 0.046\\
    \botrule
    \end{tabular*}
    \caption{Description ...}
    \label{tab:1_tab}
    \end{table}


\section{Conclusions}
\label{sec:conclusions}
Insert conclusion here... \cite{kittel_2004}, \cite{metodo_tc}


% To add bibliography (file references.bib)
\bibliographystyle{plain}
\bibliography{references}{}

\end{document}
