%% =============================================================================
%% Font setting
%% =============================================================================

%% Main font
\setmainfont{Equity-Text-A-Regular}[
    Path=\mainpath/fonts/Equity/,
    Extension=.ttf,
    UprightFont=Equity-Text-A-Regular,
    BoldFont=Equity-Text-A-Bold,
    ItalicFont=Equity-Text-A-Italic,
    BoldItalicFont=Equity-Text-A-Bold-Italic,
    SmallCapsFont=Equity-Caps-A-Regular,
    BoldFeatures = {
        SmallCapsFont=Equity-Caps-A-Bold
    }
]

%% Sans-serif
 \setsansfont{Concourse-T3-Regular}[
     Scale=MatchLowercase,
     Path=\mainpath/fonts/Concourse/,
     Extension=.otf,
     UprightFont=Concourse-T3-Regular,
     BoldFont=Concourse-T3-Bold,
     ItalicFont=Concourse-T3-Italic,
     BoldItalicFont=Concourse-T3-Bold-Italic,
     SmallCapsFont=Concourse-C3-Regular,
     BoldFeatures = {
         SmallCapsFont=Concourse-C3-Bold
     }
 ]

 %% Sans-serif small caps bold
 \newfontfamily{\scfont}{Concourse-C3-Regular}[
     Path=\mainpath/fonts/Concourse/,
     Extension=.otf,
     UprightFont=Concourse-C3-Regular,
     BoldFont=Concourse-C3-Bold,
     WordSpace=1.0,
     LetterSpace=-4
 ]

 %% Upper case font
 \newfontfamily{\ucfont}{Concourse-T3-Regular}[
     Scale=MatchUppercase,
     Path=\mainpath/fonts/Concourse/,
     Extension=.otf,
     UprightFont=Concourse-T3-Regular,
     BoldFont=Concourse-T3-Bold,
     ItalicFont=Concourse-T3-Italic,
     BoldItalicFont=Concourse-T3-Bold-Italic,
     WordSpace=2.5,
     LetterSpace=10
 ]

 %% Page number font
 \newfontfamily{\pnfont}{Concourse-T6-Regular}[
     Scale=MatchUppercase,
     Path=\mainpath/fonts/Concourse/,
     Extension=.otf,
     UprightFont=Concourse-T6-Regular,
     WordSpace=2.5,
     LetterSpace=10
 ]

 %% Mono-spaced
  \setmonofont{Inconsolata}[
      Path=\mainpath/fonts/Inconsolata/,
      Extension=.ttf,
      Scale=MatchLowercase,
      WordSpace=0.8,
      LetterSpace=0.0,
      UprightFont=Inconsolata-Regular,
      BoldFont=Inconsolata-Bold
  ]

 \fontdimen2\font=.25em % or .3333em
 \fontdimen3\font=\dimexpr.5em-\fontdimen2\font
 \fontdimen4\font=\dimexpr\fontdimen2\font-.2em


%% =============================================================================
%% Colors
%% =============================================================================

%% Links
\definecolor{ccolor}{HTML}{bc2323} % Cite
\definecolor{dcolor}{HTML}{2370bc} % DOI
\definecolor{rcolor}{HTML}{bc2323} % Internal links

%% =============================================================================
%% TikZ styles
%% =============================================================================

%% Styles
\tikzstyle{roundlabel}=[circle,inner sep=1.5pt,fill=tikzgrey,color=black,text=white,draw,font=\sffamily\footnotesize]
\tikzstyle{TlabelB}=[rectangle callout,rounded corners=0.03cm,inner sep=3.0pt,fill=tikzlight,color=tikzlight,text=tikzgrey,draw,font=\sffamily\footnotesize,callout absolute pointer={#1},at={#1},above=0.2cm,align=center]
\tikzstyle{smallbullet}=[circle,inner sep=0.5pt,fill=tikzorange,color=tikzorange,text=white,draw,font=\sffamily\small\bfseries]
\tikzstyle{labelline}=[rounded corners=0.5mm,black,line width=2pt,solid]
\tikzstyle{labeltext}=[color=white,fill=white,inner sep=0.5pt,text=tikzgrey,draw,font=\sffamily\footnotesize]
\tikzstyle{math}=[font=\Large,align=center]
\tikzstyle{figure}=[anchor=south west]
\tikzstyle{label}=[circle,inner sep=0.07cm,fill=white,text=black,draw=white]
\tikzstyle{tlabel}=[inner sep=0.07cm,text=black]
\tikzstyle{wlabel}=[inner sep=0.07cm,text=black,anchor=west,font=\footnotesize]
\tikzstyle{figtext}=[anchor=west, rounded corners=0.0cm,inner sep=0.07cm,fill=white,color=white,text=black,draw,font=\sffamily\footnotesize]

%% Grid
\newcommand{\grid}{%
    \draw[step=0.2cm,sublatticecolor,very thin] (0,0) grid (11,17);
    \draw[step=1.0cm,black,thin] (0,0) grid (11,17);
    \foreach \x in {0,1,...,11}
    \draw (\x cm,1pt) -- (\x cm,-1pt) node[anchor=north,font=\sffamily\tiny] {\x};
    \foreach \y in {0,1,...,17}
    \draw (1pt,\y cm) -- (-1pt,\y cm) node[anchor=east,font=\sffamily\tiny] {\y};
}

%% Scale (for labels)
\tikzset{every picture/.append style={scale=1.22807}}


%% =============================================================================
%% Itemize labels
%% =============================================================================

% Itemize labels
\def\labelitemi{\raisebox{0.3mm}{\hbox{\normalsize\pmb{\rightarrow}}}}

% Enumerate labels
\setenumerate[1]{%
    label={\sffamily\bfseries\arabic*},
    itemsep=0.8em,
    topsep=1em,
    labelsep=1.0em
}

\setenumerate[2]{%
    label={\sffamily\bfseries\alph*},
    itemsep=0.8em,
    topsep=1em,
    labelsep=1.0em
}

%% =============================================================================
%% Labels
%% =============================================================================

%% Figures
\newcommand{\lbl}[1]{\bf\sffamily\small(#1)}
\newcommand{\CaptionMark}[1]{{\boldmath\textbf{#1}}}
\newcommand{\CaptionLabel}[1]{\textbf{(#1)}}
\newcommand{\CaptionLabelR}[1]{(#1)}

%% Reformat equation labels at equations and in text
\makeatletter
\def\tagform@#1{\maketag@@@{\footnotesize\ignorespaces#1\unskip\@@italiccorr}}
\makeatother

%% Figure numbering
\counterwithin{figure}{chapter}
\counterwithin{table}{chapter}

%% =============================================================================
%% Math environments
%% =============================================================================

%% Spacing in cases
\makeatletter
\def\env@cases{%
    \let\@ifnextchar\new@ifnextchar
    \left\lbrace
    \def\arraystretch{1.2}%
    \array{l@{\quad}l@{}}% Formerly @{}l@{\quad}l@{}
}
\makeatother

%% =============================================================================
%% Chapter und section design
%% =============================================================================

%% Dotfill
\makeatletter
\newcommand \Dotfill {\leavevmode \cleaders \hb@xt@ .25em{\hss .\hss }\hfill \kern \z@}
\makeatother

%% Section
\titleformat{\section}[hang]
{\normalfont\raggedright\huge}
{\thesection}
{0.3cm}
{#1}
\titlespacing*{\section}{0pt}{4.5ex plus 1ex minus .2ex}{3.5ex plus .2ex}

%% Subsection
\titleformat{\subsection}[hang]
{\normalfont\Large\raggedright}
{\thesubsection}
{0.3cm}
{#1}
\titlespacing*{\subsection}{0pt}{3.5ex plus 1ex minus .2ex}{2.3ex plus .2ex}

%% Subsubsection
\titleformat{\subsubsection}[hang]
{\normalfont\large\raggedright\bfseries\boldmath}
{}
{0.0cm}
{#1}
\titlespacing*{\subsubsection}{0pt}{4.5ex}{2.3ex}

%% Paragraph
\titleformat{\paragraph}[hang]
{\normalfont\normalsize\raggedright\itshape}
{}
{0.0cm}
{#1}
\titlespacing*{\paragraph}{0pt}{4.5ex}{2.5ex}

%% Subparagraph
\titleformat{\subparagraph}[runin]
{\normalfont\normalsize\raggedright\itshape}
{}
{0.0cm}
{#1~---}
\titlespacing*{\subparagraph}{0pt}{2.5ex}{0.5ex}


%% =============================================================================
%% Aphorisms
%% =============================================================================

\makeatletter
\newcommand{\aphorismMOD}[4][\@nil]{
    \def\aphtmp{#1}%
    \hfill
    \parbox{#4}{
        \parbox{\linewidth}{
            \raggedright%
            \setstretch{1.1}%
            {\footnotesize%
                \textit{#2}\par}
            }
            \\[12pt]
            \parbox{\linewidth}{
                \raggedright%
                \setstretch{1.1}%
                {\footnotesize%
                    ---~\textsc{#3}%
                    \ifx\aphtmp\@nnil\else%
                    \\\leavevmode\phantom{---}~\textit{in}\enspace{#1}
                    \fi%
                \par}
            }
        }
        \\[12mm]
    }
    \makeatother


%% =============================================================================
%% Listings
%% =============================================================================

\lstset{
        basicstyle=\scriptsize\sffamily, % Standardschrift
        % numbers=left,               % Ort der Zeilennummern
        numberstyle=\tiny\sffamily, % Stil der Zeilennummern
        stepnumber=1,               % Abstand zwischen den Zeilennummern
        numbersep=0pt,              % Abstand der Nummern zum Text
        tabsize=2,                  % Groesse von Tabs
        extendedchars=true,         %
        breaklines=true,            % Zeilen werden Umgebrochen
        keywordstyle=\color{tikzorange},
        % frame=l,
        % framerule=1.5pt,
        % rulecolor=\color{black},
      % keywordstyle=[1]\textbf,    % Stil der Keywords
      % keywordstyle=[2]\textbf,    %
      % keywordstyle=[3]\textbf,    %
      % keywordstyle=[4]\textbf,   \sqrt{\sqrt{}} %
        stringstyle=\bfseries\sffamily, % Farbe der String
        showspaces=false,           % Leerzeichen anzeigen ?
        showtabs=false,             % Tabs anzeigen ?
        xleftmargin=17pt,
        framexleftmargin=5pt,
        framexrightmargin=5pt,
        framextopmargin=5pt,
        framexbottommargin=5pt,
        % framesep=10pt,
      %framexbottommargin0pt,
        backgroundcolor=\color{listingback},
        showstringspaces=false      % Leerzeichen in Strings anzeigen ?
    }

\lstloadlanguages{% Check Dokumentation for further languages ...
      % [Visual]Basic
      % Pascal
      % C
        C++
      % XML
      % HTML
      %Java
    }

\DeclareCaptionFont{white}{\color{white}}
\DeclareCaptionFormat{listing}{\colorbox{mygray}{\parbox{0.988\linewidth}{\hspace{5pt}#1#2#3}}}

%% =============================================================================
%% Hyperref
%% =============================================================================

\hypersetup{
        %bookmarks=true,        % show bookmarks bar?
        unicode=false,          % non-Latin characters in Acrobat’s bookmarks
        pdftoolbar=true,        % show Acrobat’s toolbar?
        pdfmenubar=true,        % show Acrobat’s menu?
        pdffitwindow=false,     % window fit to page when opened
        pdfstartview={FitH},    % fits the width of the page to the window
        pdftitle={}, % title
        pdfauthor={Nicolai Lang},       % author
        pdfsubject={Dissertation 2019},   % subject of the document
        pdfcreator={Nicolai Lang},      % creator of the document
        pdfproducer={Nicolai Lang},     % producer of the document
        pdfkeywords={}, % list of keywords
        pdfnewwindow=true,      % links in new window
        colorlinks=true,        % false: boxed links; true: colored links
        linkcolor=rcolor,      % color of internal links
        citecolor=ccolor,         % color of links to bibliography
        filecolor=magenta,      % color of file links
        urlcolor=dcolor           % color of external links
    }

    %% Font for URLs
    \renewcommand\UrlFont{\ttfamily}

    %% Backref
    \renewcommand*{\backref}[1]{}
    \renewcommand*{\backrefalt}[4]{%
        \ttfamily\footnotesize
        \ifcase #1 (Not cited.)%
        \or        (Cited on page~#2.)%
        \else      (Cited on pages~#2.)%
    \fi}


%% =============================================================================
%% Captions
%% =============================================================================

%% Styling
\DeclareCaptionLabelSeparator{vline}{~•~}
\DeclareCaptionFormat{myformat}{\small\sffamily{\scfont\textbf{#1#2}}#3}

\DeclareCaptionStyle{mystyle}[
        margin=5mm,
        justification=centering
    ]{margin={0mm,0mm},labelsep=vline}

\captionsetup{style=mystyle,format=myformat}

    %% 'See next page' marker
\newcommand{\contnextpage}{~~$\curvearrowright$}
\newcommand{\continued}{$\curvearrowright$~~}


%% =============================================================================
%% Header & Footer design
%% =============================================================================

\makeatletter

\renewcommand{\headrule}{
                    %\vbox to 7pt{\hbox to \linewidth{\color{top}\footnotesize\dotfill}\vss}
}

\def\cleardoublepage{
    \clearpage\if@twoside \ifodd\c@page\else
    \hbox{}
    \thispagestyle{empty}
    \newpage
    \if@twocolumn\hbox{}\newpage\fi\fi\fi
}

\makeatother



%% -----------------------------------------------------------------------------
%% Header layout for A4 (digital)
%% -----------------------------------------------------------------------------

%% Main
\fancypagestyle{default}{
    \fancyhf{}
    \fancyfoot[CO]{\pnfont\footnotesize\thepage}
    \fancyfoot[CE]{\pnfont\footnotesize\thepage}
    \fancyhead[CO]{\ucfont\footnotesize\MakeUppercase{\rightmark}}
    \fancyhead[CE]{\ucfont\footnotesize\MakeUppercase{\leftmark}}
}

%% Summary
\fancypagestyle{summary}{
    \fancyhf{}
    \fancyfoot[CE]{\pnfont\footnotesize\thepage}
    \fancyfoot[CO]{\pnfont\footnotesize\thepage}
    \fancyhead[CE]{\ucfont\footnotesize\MakeUppercase{In a nutshell}}
    \fancyhead[CO]{\ucfont\footnotesize\MakeUppercase{In a nutshell}}
}

%% Numbered
\fancypagestyle{numbered}{
    \fancyhf{}
    \fancyfoot[CE]{\pnfont\footnotesize\thepage}
    \fancyfoot[CO]{\pnfont\footnotesize\thepage}
}

%% Plain page empty
\fancypagestyle{plain}{
    \renewcommand{\headrule}{}
    \fancyhf{}
    \fancyfoot{}
    \fancyhead{}
}

%% Set chapter and section marks
\pagestyle{default}
\renewcommand{\chaptermark}[1]{\markboth{#1}{}}
\renewcommand{\sectionmark}[1]{\markright{#1}{}}


%% =============================================================================
%% Footnotes
%% =============================================================================

%% Running numbering
\makeatletter \@removefromreset{footnote}{chapter} \makeatother


%% =============================================================================
%% Typography
%% =============================================================================

%% Solid end-of-proof symbol
\renewcommand{\qedsymbol}{$\blacksquare$}

%% Spacing between equations and text
\makeatletter
\g@addto@macro\normalsize{%
    \abovedisplayskip=10.0pt plus 1.0pt minus 1.0pt
    \belowdisplayskip=10.0pt plus 1.0pt minus 1.0pt
    \abovedisplayshortskip=0.0pt plus 1.0pt minus 1.0pt
    \belowdisplayshortskip=6.0pt plus 1.0pt minus 1.0pt
}
\makeatother

%% Avoid orphans etc.
% \clubpenalty = 10000
% \widowpenalty = 10000
% \displaywidowpenalty = 10000

%% DOI style
\let\origdoi\doi
\renewcommand{\doi}[1]{{\footnotesize\bfseries\texttt{\origdoi{#1}}}}

%% arXiv
\newcommand{\arxiv}[1]{{\footnotesize\bfseries\texttt{arXiv:\href{http://arxiv.org/abs/#1}{#1}}}}
\let\eprint\arxiv

%% =============================================================================
%% Theorems
%% =============================================================================

\newtheoremstyle{theoremitalic}% name of the style to be used
{20pt}% measure of space to leave above the theorem. E.g.: 3pt
{20pt}% measure of space to leave below the theorem. E.g.: 3pt
{\itshape}% name of font to use in the body of the theorem
{0pt}% measure of space to indent
{}% name of head font
{}% punctuation between head and body
{ }% space after theorem head; " " = normal interword space
{\sffamily\textbf{\textsc{#1~#2}:\enspace#3}\\*[5pt]}

\newtheoremstyle{theoremsans}% name of the style to be used
{20pt}% measure of space to leave above the theorem. E.g.: 3pt
{20pt}% measure of space to leave below the theorem. E.g.: 3pt
{\sffamily}% name of font to use in the body of the theorem
{0pt}% measure of space to indent
{}% name of head font
{}% punctuation between head and body
{ }% space after theorem head; " " = normal interword space
{\sffamily\textbf{\textsc{#1~#2}:\enspace#3}\\*[5pt]}

\theoremstyle{theoremitalic}
\newtheorem{myresult}{Result}[chapter]
\newtheorem{myprop}{Proposition}[chapter]
\newtheorem{mylemma}{Lemma}[chapter]
\newtheorem{mytheorem}{Theorem}[chapter]
\newtheorem{myconjecture}{Conjecture}[chapter]
\newtheorem{mycorollary}{Corollary}[chapter]
\newtheorem{mydef}{Definition}[chapter]
\newtheorem{myremark}{Remark}[chapter]

\theoremstyle{theoremsans}
\newtheorem{myexample}{Example}[chapter]


%% =============================================================================
%% Appendices
%% =============================================================================

%% Appendices after each chapter
\AtBeginEnvironment{subappendices}{%
    {%
        \titleformat{\chapter}[display]%
        {\normalfont\scshape\bfseries}%
        {}{0cm}{%
            \centering\Large #1 for Chapter\enspace{\Huge\thechapter}\\
            % \rule{\textwidth}{4pt}
        }
        \titlespacing*{\chapter}{0pt}{10ex}{12ex}
        \chapter*{Appendices}
        \addcontentsline{toc}{chapter}{\protect\hphantom{\numberline{\thechapter}}\normalfont\textit{Appendices}}%
    }
    % \counterwithin{figure}{section}
    % \counterwithin{table}{section}
}



%% =============================================================================
%% Floats
%% =============================================================================

%% Place float at even pages
\newcommand\atevenpage[1]{%
    \afterpage{\clearpage% be sure, that there are no pending floats
        \ifodd\value{page}% still a odd page
        \atevenpage{#1}%
        \else
        #1%
        \fi
    }%
}

%% Place float at odd pages
\newcommand\atoddpage[1]{%
    \afterpage{\clearpage% be sure, that there are no pending floats
        \ifeven\value{page}% still a odd page
        \atoddpage{#1}%
        \else
        #1%
        \fi
    }%
}

%% Floats get their own page
\renewcommand{\topfraction}{0.85}
\renewcommand{\floatpagefraction}{.85}
\renewcommand{\textfraction}{.15}

%% Flush figures to top of page
\makeatletter
\setlength{\@fptop}{0pt}
\setlength{\@fpbot}{0pt plus 1fil}
\makeatother


%% =============================================================================
%% Namespaces (make labels local)
%% =============================================================================

\newcommand{\usingnamespace}[1]{\def\namespace{#1}}

\AtBeginDocument{%
    \newcommand\id[1]{\label{\namespace:#1}}
    \newcommand\idref[1]{\cref{\namespace:#1}}
    \newcommand{\idrefrange}[2]{\crefrange{\namespace:#1}{\namespace:#2}}
    \newcommand\Idref[1]{\Cref{\namespace:#1}}
    \newcommand{\Idrefrange}[2]{\Crefrange{\namespace:#1}{\namespace:#2}}

    %% Autoref
    % \renewcommand{\chapterautorefname}{Chapter}
    % \renewcommand{\sectionautorefname}{Section}
    % \renewcommand{\subsectionautorefname}{Subsection}
    % \renewcommand{\figureautorefname}{Figure}
    % \renewcommand{\tableautorefname}{Table}
    % %% My environments
    % \newcommand{\lemmaautorefname}{Lemma}
}

%% Cleveref
\crefname{equation}{eq.}{eqs.}
\Crefname{equation}{Eq.}{Eqs.}
\crefname{figure}{fig.}{figs.}
\Crefname{figure}{Fig.}{Figs.}
\crefname{table}{tab.}{tabs.}
\Crefname{table}{Tab.}{Tabs.}
\crefname{section}{sec.}{secs.}
\Crefname{section}{Sec.}{Secs.}
\crefname{subsection}{subsec.}{subsecs.}
\Crefname{subsection}{Subsec.}{Subsecs.}
\crefname{chapter}{chapter}{chapters}
\Crefname{chapter}{Chapter}{Chapters}
\crefname{footnote}{footnote}{footnotes}
\Crefname{footnote}{Footnote}{Footnotes}
\crefname{mytheorem}{theorem}{theorems}
\Crefname{mytheorem}{Theorem}{Theorems}


%% =============================================================================
%% Table of contents
%% =============================================================================

%% Fix width of page number column in TOC
\makeatletter
\renewcommand{\@pnumwidth}{6.5mm}
\makeatother


%% =============================================================================
%% Index
%% =============================================================================

\newcommand{\IDX}[1]{#1\index{#1}}

\newcommand{\keywords}[1]{%
    \renewcommand*{\do}[1]{\index{##1|(}}
    \docsvlist{#1}
}

\newcommand{\delkeywords}[1]{%
    \renewcommand*{\do}[1]{\index{##1|)}}
    \docsvlist{#1}
}

%% =============================================================================
%% Formatting
%% =============================================================================

\newcommand{\cpdraft}{%
    % \ifdraft\clearpage\fi
    \clearpage
}

\newcommand{\pbdraft}[1]{%
    % \ifdraft\pagebreak[#1]\fi
    \pagebreak[#1]
}

\newcommand{\npbdraft}[1]{%
    % \ifdraft\nopagebreak[#1]\fi
    \nopagebreak[#1]
}
